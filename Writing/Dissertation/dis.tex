\documentclass[oldfontcommands]{memoir}
\usepackage{chicago}

%\renewenvironment{thebibliography}[1]{\bibsection}{}
\renewenvironment{thebibliography}[1]{%
  \bibsection
  \begin{bibitemlist}{xxxxxxxxxx}}{\end{bibitemlist}\postbibhook}

\renewcommand{\maketitlehookc}{\begin{center} Some stuff \end{center}}
\newcommand{\myepigraph}[3]{\epigraph{#1}{\textit{#2}\\ \textsc{#3}}}

\newcommand{\sequenceIndexPageFontSolved}[1]{{\textbf #1}}
\newcommand{\sequenceIndexPageFont}[1]{{\textit #1}}
\newcommand{\sequence}[1]{{\ensuremath #1\ldots}\index[sequences]{#1@{\ensuremath #1\ldots}|sequenceIndexPageFont}}
\newcommand{\sequenceSolved}[1]{{\ensuremath #1\ldots}\index[sequences]{#1@{\ensuremath #1\ldots}|sequenceIndexPageFont}}


\makeindex[sequences]

\begin{document}

\frontmatter

\title{Seqsee: \\ A cognitive architecture}
\author{Abhijit A. Mahabal}
\date{Mar 2008}
%\thispagestyle{empty}
\maketitle

\clearpage
\tableofcontents

\onecolindextrue
\renewcommand{\indexname}{Index of Sequences}
\renewcommand{\preindexhook}{These are the sequences that have been used in the dissertation. Page numbers like \sequenceIndexPageFont{7} indicate a discussion related to the sequence, whereas page numbers set like \sequenceIndexPageFontSolved{7} indicate a mention that the program ``solved'' it.\par\vskip0.5in}
\printindex[sequences]
\renewcommand{\preindexhook}{}

\mainmatter
%\chapterstyle{companion}
\pagestyle{Ruled}

\chapter{Some Introduction}
\myepigraph{All men are born equal, but some are more equal than others.}{Animal Farm}{George Orwell}

What happens on tjr next page?
\sequence{1 2 3 4 5 6}

\newpage

\cite{Mitchell90} and \citeN[some cmt]{Hofstadter:FCCA}

And more stuff \sequence{1 2 3 4 5 6} and also \sequence{1 2 3 4 5 6 7} and \sequenceSolved{ 1 2 3 4 5 6}

\bibliographystyle{chicago}
\bibliography{merged}
\end{document}
