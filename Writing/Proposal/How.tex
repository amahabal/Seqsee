%\documentclass{article}
%\usepackage{chicago}
%\begin{document}

\section{How Seqsee works currently}
\label{sec:how}

What follows is a broad brush stroke description of the working of Seqsee and it's theoretical underpinnings.  For want of space it skimps on details but I would be happy to elaborate on any of this during the actual presentation.

\subsection{The workspace and the codelets}

Seqsee, like copycat and other FARG projects, has a workspace which is the arena where ``perceived objects'' live.  These perceived objects include chunks that Seqsee has created out of the input elements, the relations between various objects other things besides.  The workspace is almost exactly the same as its namesake in Copycat and I shall not comment on it further.

Codelets, too, are similar to their Copycat counterpart. Each codelet contains a small piece of code that does some microscopic act of cognition.  Examples include checking if two objects are related and checking if an object belongs to a certain category.

The only different thing in Seqsee is a variety of codelets that can be run immediately rather than being stored on the coderack.  This can of course be simulated by the Copycat codelet system by using codelets of extremely high urgency (thereby causing that codelet to get run almost immediately).

\subsection{The discovery of similarity}

The codelet \emph{find-if-related} can look at two objects and determine---given its current knowledge of the two objects---if it can find a relation between the two.  This section is \emph{not} its story.  Rather, it is the story of why some other part of Seqsee created this codelet to work on those two specific objects.  In other words, speaking anthropomorphically, why somebody thought that these two objects had some chance of being related.

Here Seqsee diverges from Copycat.  Copycat also has a codelet with a similar function: the \emph{bond-scout}.  The difference is that the \emph{bond-scout} chooses it's objects randomly (though preferring salient objects).  Even the top-down cousin of the \emph{bond-scout} chooses its objects randomly.  \emph{find-if-related} on the other hand gets called only on objects that could be ``similar''.  This has a major advantage in terms of how smoothly the discovery of structures proceeds.  (Several other things in Seqsee are bottom-up, of course).

The story unravels in three very related segments.

\subsubsection{The fringe of an object}
\label{sec:fringe}

The notion that concepts have a fringe of floating associations is not new, going back to at least William James \cite{James}.  Several people have commented on the fringe, including Hofstadter.  For instance, Chapter 10 \emph{On words and their magical halos} of \citeN{Hofstadter:LeTon} deals with this.  A particular example from this chapter is that of the concept \emph{cheese}, and how it brings up images of either ``orangy processed squares that come pre-sliced and pre-wrapped in plastic'' or ``a tray full of unprocessed cheeses, perhaps a hunk of Brie, some Gouda, some che\'vre'' depending on the context.

``Concepts'' in Seqsee have fringes, and these too are context dependent.  Note that each object in the workspace is also a concept.  If you see a few numbers written on a white-board, then you may point to the first \emph{7} on the board.  That \emph{7} is a concept, different from another \emph{7} on the same white-board.  These are temporary concepts in that they would almost certainly not enter the long-term memory, and similarly objects in the workspace are also quiet transient.  In Seqsee, a concept activates other concepts to various degrees and thus other concepts are in the fringe to different degrees. Getting back to the story, even objects in the workspace may have fringes.

I will hold on for the moment to what fringes contain, but in short two concepts are considered worth the attention if their fringes overlap.  This can be made concrete with the following example.  In the sequence \emph{1 7 2 8} the fringe of the object \emph{7} contains the number seven and to a lesser extent the numbers eight and six.  The fringe of the object \emph{8} also contains the number eight and thus the two objects stand a chance of being related.

\subsubsection{Tagging}
\label{sec:tagging}

The story for \emph{7} and \emph{8} above was simple.  But how are the groups \emph{123} and \emph{1234} seen to be potentially similar?  The answer is that when they have both been perceived as \emph{ascending groups}--- and not before---they appear similar to Seqsee.  These tags are part of the fringe, and clearly, if both objects share a tag their fringes overlap.  The overlap is a matter of degrees and hence so is this feeling of similarity.

I think there is something cognitively interesting going on here that I would like to draw your attention to via a quote from Andy Clark \cite{Clark:MindWare}.

\begin{quote}
Learning a set of tags and labels is rather closely akin to acquiring a new perceptual modality.  For like a perceptual modality, it renders certain features of our world concrete and salient, and allows us to target our thoughts (and learning algorithms) on a new domain of basic objects.  This new domain compresses what were previously complex and unruly sensory patterns into simple objects.  The simple objects can then be attended to in ways that quickly reveal further (otherwise hidden) patterns, as in the case of relations between relations.
\end{quote}


In the case when the overlap of the fringes includes such tags there is more information to discover the relationship.  Seqsee does not deal with prime numbers, but pretend for a moment that it does and that it has been presented with the sequence ``2 3 5 7 11''. If both 5 and 7 have been tagged as primes then in the process of relation finding the program may discover that 7 is the next prime after 5, and use this information to understand the sequence.


\subsection{Affordances}
\label{sec:affordances}

When we go to a restaurant we know what to do.  This was the central example that Schank uses to introduce scripts \cite{Schank+Abelson}.  According to this theory, we have scripts (think acting) that help us schedule actions. 

 Another theory that attempts to explain how we know what to do is the affordences theory of Gibson. A chair affords sitting, for example, and that is how we know what to do with chairs.  The two theories are not quite at odds with each other: they are just trying to explain similar phenomena at different granularities.

The micro decisions Seqsee makes in choosing what codelets should be created next occurs via an affordences-like mechanism. Groups afford extending, for example.  Even meta thoughts afford.  The thought \emph{I am in a rut} affords destroying weak groups and trying less likely possibilities like exploring if the sequence is interlaced.

Seqsee does not currently use scripts.  It is conceivable that something similar may be called for. 

\subsection{The Stream}
\label{sec:stream}

Where all this comes together is the stream of thought.  It is the star player in the central cognitive loop.  Seqsee has objects (in the workspace) and thoughts about those objects in the stream.  Unlike the highly parallel buzz of activity of the codelets, there is something slightly more linear about the stream.  When I describe the stream, people sometimes have the sense of one thought leading to the next leading to the next---the $n+1^{\mathrm{th}}$ thought being caused somehow by the $n^\mathrm{th}$ thought.  That is not how Seqsee's stream works.  I must describe it more carefully here.

One way to start describing it is to say where the $n+1^{\mathrm{th}}$ thought can come from.  It can of course come  from the immediate prior thoughts but more importantly it can come from any codelet whatsoever.  Consider the hypothetical case where seqsee is solving five problems in parallel.  There are several codelets toiling away on each problem.  A thought in the stream corresponds to focusing attention on something for a short period of time.  In principle it could happen that the thoughts that pass through the stream are perfectly interlaced: the first, the sixth, the 11th thoughts coming from the first problem and so forth.  In this sense, then, the stream is just a list of all thoughts happening anywhere in the system, just sorted by time.  So much for linearity.

If the stream is almost merely a birth-registery of attention-fixations, what, if anything, does it buy us? What it gives us is temporality.  Seqsee keeps in memory the fringes of the last few thoughts, and these decay with time.  If the fringe of a new thought intersects with the fringe of one of the more recent thoughts from anywhere in the system, Seqsee's similarity-ears get perked.  In the hypothetical five problem case we would see very strong recency effects and interference from seqsee.  It is as if the stream is distributed throughout the "brain" of seqsee, but whenever a thought occurs anywhere its echo reverberates throughout, potentially influencing for a short time the processing everywhere.

This is not a completely unmotivated move on my part.  A somewhat similar mechanism is used by \citeN{Dennett:Consciousness} in what he calls the \emph{Joycean machine}---a more or less serial machine simulated imperfectly on parallel neural hardware.

More pragmatically, using the stream has made seqsee much smarter (as compared to Seqsee 12 months ago when there was no stream).  In a sequence like \emph{1 1 1 1 17 2 2 2 2 18} the way Seqsee discovers the 17-18 relationship is by being lucky and thinking about the 17 and 18 somewhat close together in time.  Because of all the hubbub of the parallel codelet activity such lucky episodes are guaranteed to happen.  Without the stream, seeing the 17 -- 18 relationship was much more unlikely.  With a \emph{bond-scout} like mechanism where objects are randomly chosen it is necessary to bias against choosing objects distant from each other because such random long-distance checking is statistically less likely to produce useful results.  So even though the 17-18 relationship stands a chance of being discovered it is not predestined.
