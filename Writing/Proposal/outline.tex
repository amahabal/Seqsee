\documentclass{article}
\makeatletter
\renewcommand{\theenumi}{\Roman{enumi}}
\renewcommand{\theenumii}{\Alph{enumii}}
\renewcommand{\theenumiii}{\alph{enumiii}}
\renewcommand{\theenumiv}{\roman{enumiv}}
\makeatother
\begin{document}

\begin{enumerate}
\item Introduction
\item Related work
  \begin{enumerate}
  \item FARGonauts \cite{Mitchell90} \cite{Hofstadter:FCCA Chapter 1}
  \item Similar task: \cite{SimonKotovsky, Meredith, Laird, Persson, Pivar}
  \item Unrelated work: extrapolation
  \item Relevant cognitive ideas:
    \begin{enumerate}
    \item labeling
    \item fringe
    \item Meta cognition, ``Control''
    \item Affordances
    \end{enumerate}
  \item How ACT-R may be construed to be ``related work''
  \end{enumerate}
\item The task in greater detail
  \begin{enumerate}
  \item The domain
    \begin{enumerate}
    \item Patterns, not math. Examples, and some non examples
    \end{enumerate}
  \item Some annotated should-see sequences
  \item More than just extending
    \begin{enumerate}
    \item Creating variations
    \item Memory (a l\'a Metacat)
    \item Meta-thoughts
    \end{enumerate}
  \item Subtleties
    \begin{enumerate}
    \item What constitutes understanding?
      \begin{enumerate}
      \item Seeing motion, hitting walls?
      \end{enumerate}
    \item Multiple parsings
    \end{enumerate}
  \end{enumerate}
\item Framework and Methodology
\item How Seqsee works (currently)
  \begin{enumerate}
  \item Intro; explain necessarily sketchy
  \item The workspace and codelets
    \begin{enumerate}
    \item No deviation yet from copycat
    \item Justifying: quote \cite{Hofstadter:FCCA}, \cite{Hearsay}
    \end{enumerate}
  \item The ``discovery'' of similarity
    \begin{enumerate}
    \item The FindIfRelated codelet and how it gets triggered. The fringe \cite{James}
    \item Tagging; Justification using \cite{Clark:MagicWords} and \cite{Dennett:Labeling}
      \item Tags, fuzzy categories and analogies: \cite{Hofstadter:LeTon}
    \end{enumerate}
    \item How thought leads to thought. Affordances.
    \item Puttint it all together
      \begin{enumerate}
      \item The stream of thought.
      \item A far from perfect metaphor:
        \begin{enumerate}
        \item n+1 not from n
        \item random other processes can add to it
        \item A better imagery is just ``thoughts sorted by time''
        \item In a limiting case, like \cite{Dennett:Consciousness}'s stream.
        \end{enumerate}
      \item How fringes and affordances work.
      \end{enumerate}
  \end{enumerate}
  
\item Timeframe
\item References

\end{enumerate}

\bibliographystyle{apalike}
\bibliography{merged}

\end{document}
