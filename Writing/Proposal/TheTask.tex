\documentclass{article}

\begin{document}

\section{The task in greater detail}
\label{sec:task}


\subsection{The domain}
\label{sec:domain}

Seqsee, as mentioned earlier, had a predecessor named Seek-Whence \cite{Meredith}. But Hofstadter's interest in sequence-extending programs goes back further. In the original pre-Seek-Whence guise, a typical input for the program could have been, for instance, the sequence of alternate primes \emph{2 5 11 17 23$\ldots$}.

Gradually Hofstadter's interest shifted away from such mathematical-knowledge intensive sequences to a smaller domain: that of sequences which involve no knowledge of arithmetics beyond the relationship of successor and predecessor, and the ability to count. This meant outlawing such sequences as the primes, and even the Fibonacci sequence as it involves addition. The shift to this austere domain is described in \cite[pages 48-49]{Hofstadter:FCCA}.

This restricted domain is by no means poor in sequences that tax the intellect, or even in sequences excellent for exploring program-cognition. For sequence k in the next section, for example, it is interesting to contemplate how a program might navigate the \emph{2 2 2 2 2} conundrum. Several other sequences require for their honest human-like understanding the ability to chunk (almost all examples in the next section) and even to see an object \emph{as} another.

I shall not here try to delineate further the problems that could be the inputs for Seqsee, but leave that task instead to the dozen or so examples in the next section.

\subsection{A wish-list of sequences}
\label{sec:list}

The following annotated listing of sequences is offered in ordered to add accountability: These are sequences that Seqsee should ``get'' before we can pronounce it \emph{good enough}. The list starts out rather simply, but builds tempo as it goes along.

Most of these sequences come from \cite[Chapter 1]{Hofstadter:FCCA}.

\renewcommand{\theenumi}{\alph{enumi}}
\begin{enumerate}
\item \textbf{1 2 3 4 $\ldots$}

Seqsee must get this, of course, but it should also get this rapidly even if a hundred terms are initially presented. It's performance should be about linear. If it takes time quadratic (or even just super-linear) in the number of input terms, there is something seriously wrong with the model.

\item \textbf{ 1 1 2 2 3 3 $\ldots$}
\item \textbf{ 1 2 2 3 3 3 $\ldots$}

Seqsee should see the initial ``group of size 1''.

\item \textbf{12 13 14 11 12 13 14 10 11 12 13 14 $\ldots$}

The program---and it is aware of negative numbers---should know that it is going to hit a wall in the future.

\item \textbf{1 1 2 3 1 2 2 3 4 1 2 3 3 4 5 $\ldots$}

I want Seqsee to be \emph{able} to see the initial 1123 as a variant of 123, and likewise the following groups as variants of 1234 and 12345.

\item \textbf{1 1 2 3 1 2 2 3 1 2 3 3 1 1 2 3 1 2 2 3 1 2 3 3$\ldots$}
\item \textbf{1 1 2 3 1 2 2 3 1 2 3 3 1 2 2 3 1 1 2 3 1 2 2 3$\ldots$}
\item \textbf{1 1 2 3 1 2 2 3 1 2 3 3 1 2 3 1 2 3 1 2 3$\ldots$}

These three sequences are all ``doubler'' sequences: the first two were called marching- and bouncing-doubler, respectively, in \cite{Hofstadter:FCCA}. I call the last the piercing-doubler.

For these, Seqsee should be able to, as in sequence e above, an endless list of 123 variants, with funny things happening. As the names of these sequences, people can perceive ``motion'' happening in these, and I'd like the program to be able to, too. It should also notice that all three are similar sequences, cut from the same cloth.

\item \textbf{ 2 10 1 3 11 2 4 12 3$\ldots$}
\item \textbf{ 2 10 2 3 11 2 3 4 12$\ldots$}

These interlaced sequences cannot be perceived merely by a rudimentary ``try every n$^\textrm{th}$ term and see if they fit together'', as can be seen from sequence j.

\item \textbf{ 2 1 2 2 2 2 2 3 2 2 4 2$\ldots$}

A confounding cousin of i and j above.

\item \textbf{ 1 0 0 1 1 1 0 0 0 0$\ldots$}
\item \textbf{ 1 2 3 1 2 2 3 1 2 2 2 3$\ldots$}
\item \textbf{ 5 4 3 4 5 4 3 4 5 $\ldots$}
\item \textbf{ 1 2 3 3 4 4 5 5 5 6 6 6$\ldots$}

Somehow I feel that the last one above is going to require the greatest amount of work.
\end{enumerate}

\subsection{More than just extending}
\label{sec:more}


\subsection{Subtleties}
\label{sec:subtleties}

\bibliographystyle{apalike}
\bibliography{merged}

\end{document}
