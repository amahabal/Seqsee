\documentclass{article}
\usepackage{xspace}
\usepackage{makeidx}
\usepackage{relsize}
\usepackage{chicago}

\makeindex
\begin{document}

\title{Seqsee: Yet Another Cognitive Architecture\\{\relsize{-3}(Ph.D. Proposal)}}
\author{Abhijit Mahabal}
\date{Oct 6, 2006}
\maketitle

\renewcommand{\baselinestretch}{1.5}\small

\section*{Introduction}
\label{sec:intro}

\newcommand{\V}[1]{version-#1\xspace}
\newcommand{\hof}{Hofstadter\xspace}
\newcommand{\dan}{Dan Dennett\xspace}
\newcommand{\andy}{Andy Clark\xspace}
\newcommand{\seq}{Seqsee\xspace}
\newcommand{\tofillout}[1]{[Need to fill gap: #1\index{Gaps!#1}]}


My Ph.D. project may be looked at from two related but quite distinct perspectives.  The first view which I call \V{A} could seem to a distant observer to be all that my project is about:

\paragraph{Version A.} I am designing and implementing a computer system that extends pattern sequences in human-like ways.

\paragraph{} The second view---\V{B}---is closer to my main interest in the project:

\paragraph{Version B.} Simulating how we think and all the spell-it-out that such an endeavor entails is likely to bring to the fore problems and issues that can otherwise be overlooked, and writing the program, thus, is merely a cover for  exploring deeper.

\paragraph{} There is also a third version that is important to me, but which I wish not to tie to my Ph.D.:

\paragraph{Version C.}  I am designing and implementing a system to solve hard problems not easily amenable to traditional computer science, by reverse engineering, as it were, some aspects of the human mind.  Seqsee is but a test case.

\paragraph{} My reasons for not making this a requirement is that in order to claim any generality of this engineering approach I would need to show that it works on other things besides sequences.  Modifying the program to work for other cases would not be tough theoretically (hopefully), but it would still be a lot of work.

\tableofcontents

\section{The context of Seqsee}
\label{sec:context}

This section describes other work that is similar either in architecture, goals or in idea-space. This work is being carried out in the Fluid Analogies Research Group, and is highly influenced by \hof and other FARGonauts. In methodology, in architecture, and in the general types of issues tackled it is strikingly similar to Copycat and other FARG projects. It draws most substantially from \citeN{Mitchell90}, \citeN{Marshall} and \citeN{Hofstadter:FCCA}. Chapter 1 of this last cited book describes the project of sequence extension and the evolution of the project's goals.

A larger body of work within which the various FARG implementations lie is that of programs inspired by the blackboard system designed by \citeN{Reddy}, which in turn can be said to belong to systems with a pandemonium style architecture \cite{Dennett:Consciousness}, characterized by large tasks being achieved by large numbers of small relatively independent not-very-smart agents. 

This is by no means the first program to attempt this task. There have been several other projects which have shared the \V{A} goals of \seq. Most notably, Simon and Kotovsky worked on such a program. \tofillout{SimonKotovsky}\tofillout{Meredith}

As \seq is intended to extend sequences in ``a human-like way'', it's construction has been influenced by several cognitive scientists' ideas:

\begin{itemize}
\item The program gets its traction from tagging various intermediate objects with labels. These labels help \seq notice high-level similarity between constructs.

Several people including \dan and \andy have described this role of ``words'' in human language, and how it aids cognition. The specific labels that I use are fluid categories of the sort described by \hof.

I will have much more to say about this tagging in a later section.

\item Noticing similarity between constructs also happens when the ``fringe'' of a construct overlaps another. \cite{James} had described 
\end{itemize}

% \documentclass{article}

% \begin{document}

\section{The task in greater detail}
\label{sec:task}


\subsection{The domain}
\label{sec:domain}

Seqsee had a predecessor named Seek-Whence written by \citeN{Meredith}. But Hofstadter's interest in sequence-extending programs goes back further. In the original pre-Seek-Whence guise, a typical input for the program could have been, for instance, the sequence of alternate primes \emph{2 5 11 17 23$\ldots$}.

Gradually Hofstadter's interest shifted away from such mathematical-knowledge intensive sequences to a smaller domain: that of sequences which involve no knowledge of arithmetics beyond the relationship of successor and predecessor, and the ability to count. This meant outlawing such sequences as the primes, and even the Fibonacci sequence as it involves addition. The shift to this austere domain is described in \citeN[pages 48-49]{Hofstadter:FCCA}.

This restricted domain is by no means poor in sequences that tax the intellect, or even in sequences excellent for exploring program-cognition. For sequence k in the next section, for example, it is interesting to contemplate how a program might navigate the \emph{2 2 2 2 2} conundrum. Several other sequences require for their honest human-like understanding the ability to chunk (almost all examples in the next section) and even to see an object \emph{as} another.

I shall not here try to delineate further the problems that could be the inputs for Seqsee, but leave that task instead to the dozen or so examples in the next section.

\subsection{A wish-list of sequences}
\label{sec:list}

The following annotated listing of sequences is offered in ordered to add accountability: These are sequences that Seqsee should ``get'' before we can pronounce it \emph{good enough}. The list starts out rather simply, but builds tempo as it goes along.

Most of these sequences come from \cite[Chapter 1]{Hofstadter:FCCA}.

\renewcommand{\theenumi}{\alph{enumi}}
\begin{enumerate}
\item \textbf{1 2 3 4 $\ldots$}

Seqsee must get this, of course, but it should also get this rapidly even if a hundred terms are initially presented. It's performance should be about linear. If it takes time quadratic (or even just super-linear) in the number of input terms, there is something seriously wrong with the model.

\item \textbf{ 1 1 2 2 3 3 $\ldots$}
\item \textbf{ 1 2 2 3 3 3 $\ldots$}

Seqsee should see the initial ``group of size 1''.

\item \textbf{12 13 14 11 12 13 14 10 11 12 13 14 $\ldots$}

The program---and it is aware of negative numbers---should know that it is going to hit a wall in the future.

\item \textbf{1 1 2 3 1 2 2 3 4 1 2 3 3 4 5 $\ldots$}

I want Seqsee to be \emph{able} to see the initial 1123 as a variant of 123, and likewise the following groups as variants of 1234 and 12345.

\item \textbf{1 1 2 3 1 2 2 3 1 2 3 3 1 1 2 3 1 2 2 3 1 2 3 3$\ldots$}
\item \textbf{1 1 2 3 1 2 2 3 1 2 3 3 1 2 2 3 1 1 2 3 1 2 2 3$\ldots$}
\item \textbf{1 1 2 3 1 2 2 3 1 2 3 3 1 2 3 1 2 3 1 2 3$\ldots$}

These three sequences are all ``doubler'' sequences: the first two were called marching- and bouncing-doubler, respectively, in \cite{Hofstadter:FCCA}. I call the last the piercing-doubler.

For these, Seqsee should be able to, as in sequence e above, an endless list of 123 variants, with funny things happening. As the names of these sequences, people can perceive ``motion'' happening in these, and I'd like the program to be able to, too. It should also notice that all three are similar sequences, cut from the same cloth.

\item \textbf{ 2 10 1 3 11 2 4 12 3$\ldots$}
\item \textbf{ 2 10 2 3 11 2 3 4 12$\ldots$}

These interlaced sequences cannot be perceived merely by a rudimentary ``try every n$^\textrm{th}$ term and see if they fit together'', as can be seen from sequence j.

\item \textbf{ 2 1 2 2 2 2 2 3 2 2 4 2$\ldots$}

A confounding cousin of i and j above.

\item \textbf{ 1 0 0 1 1 1 0 0 0 0$\ldots$}
\item \textbf{ 1 2 3 1 2 2 3 1 2 2 2 3$\ldots$}
\item \textbf{ 5 4 3 4 5 4 3 4 5 $\ldots$}
\item \textbf{ 1 2 3 3 4 4 5 5 5 6 6 6$\ldots$}

Somehow I feel that the last one above is going to require the greatest amount of work.
\end{enumerate}

\subsection{More than just extending}
\label{sec:more}


\subsection{Subtleties}
\label{sec:subtleties}

%\bibliographystyle{apalike}
%\bibliography{merged}

% \end{document}

%\documentclass{article}
%\usepackage{chicago}
%\begin{document}

\section{How Seqsee works currently}
\label{sec:how}

What follows is a broad brush stroke description of the working of Seqsee and it's theoretical underpinnings.  For want of space it skimps on details but I would be happy to elaborate on any of this during the actual presentation.

\subsection{The workspace and the codelets}

Seqsee, like copycat and other FARG projects has a workspace which is the arena where perceived objects live.  These ``perceived objects'' include chunks of input elements that Seqsee has created, the relations between various objects other things besides.  The workspace is almost exactly the same as its namesake in copycat and I shall not comment on it further.

Codelets, too, are similar to their copycat counterpart. Each codelet contains a small piece of code that does some microscopic act of cognition.  Examples include checking if two objects are related and checking if an object belongs to a certain category, and so forth.

The only different thing in Seqsee is a variety of codelets that can be run immediately rather than being stored in the coderack.  This can of force be simulated by the copycat codelet system by using codelets of extremely high urgency thereby causing it to get run almost immediately.

\subsection{The discovery of similarity}

The codelet ``find if related'' can look at two objects and determine--- given its current knowledge of the two objects--- if it can find a relation between the two.  This section is not its story.  Rather, it is a story of why some other part of Seqsee created this codelet to work on those two specific objects.  In other words, speaking anthropomorphically, why somebody thought that these two objects had some chance of being related.

Here Seqsee diverges from copycat.  Copycat also has a codelet with a similar function: the bond scout.  The difference is that the bond scout chooses its objects randomly (though preferring salient objects).  Even the top down cousin of the bond scout chooses its objects randomly.  ``Find if related'' on the other hand gets called only on objects that could be ``similar''.  This has a major advantage in terms of how smoothly the discovery of structures proceeds.  (Several other things in Seqsee are bottom up, of course).

The story unravels in three very related segments.

\subsubsection{The fringe of an object}
\label{sec:fringe}

The notion that concepts have a fringe of floating associations is not new, going back to at least William James.  Several people have commented on the fringe, including Hofstadter.  Chapter 10 ``on words and their magical halos'' of \cite{Hofstadter:LeTon} for instance deals with this.  A particular example from this chapter is that of cheese, and how it brings up images of either ``orangy processed squares that come pre-sliced and pre-wrapped in plastic'' or ``a tray full of unprocessed cheeses, perhaps a hunk of Brie, some Gouda, some che\'vre'' depending on the context.

``Concepts'' in Seqsee have fringes, and these too are context dependent.  Note that each object in the workspace is also a concept.  If you see a few numbers written on a white-board, then you may point to the first \emph{7} on the board.  That \emph{7} is a concept, different from another \emph{7} on the same white-board.  These are temporary concepts in that they would almost certainly not enter the long-term memory, and similarly objects in the workspace are also quiet transient.  In Seqsee, a concept activates other concepts to various degrees and thus other concepts are in the fringe to different degrees.

I will hold on for the moment to what fringes contain, but in short two concepts are considered worth the attention if their fringes overlap.  This can be made concrete with the following example.  In the sequence \emph{1 7 2 8} the fringe of the object \emph{7} contains the number seven and to a lesser extent the numbers eight and six.  The fringe of the object \emph{8} also contains the number eight and thus the two objects may be related.

\subsubsection{Tagging}
\label{sec:tagging}

The story for the seven and eight above was simple.  But how are the groups 123 and 1234 seen to be potentially similar?  The answer is that when the have both been perceived as ascending groups--- and not before--- they appear similar to Seqsee.  These tags are part of the fringe, and clearly, if both are so tagged their fringes overlap.  The overlap is a matter of degrees and hence so is this feeling of similarity.

I think that is something cognitively interesting going on here that I would like to draw your attention to via a quote from \cite{Clark:MindWare}.

\begin{quote}
Learning a set of tags and labels is rather closely akin to acquiring a new perceptual modality.  For like a perceptual modality, it renders certain features of our world concrete and salient, and allows us to target our thoughts (and learning algorithms) on a new domain of basic objects.  This new domain compresses what were previously complex and unruly sensory patterns into simple objects.  The simple objects can then be attended to in ways that quickly reveal further (otherwise hidden) patterns, as in the case of relations between relations.
\end{quote}


In the case when fringes overlap includes such tags there is more information to discover the relationship.  Seqsee does not deal with prime numbers, pretend for a moment that it does and has been presented with the sequence ``2 3 5 7 11''. If both 5 and 7 have been tagged as primes then in the process of relation finding the program may discover that 7 is the next prime after 5, and use this information to understand the sequence.


\subsection{Affordances}
\label{sec:affordances}

When we go to a restaurant we know what to do.  This was the central example that Schank uses to introduce scripts \cite{Schank+Abelson}.  According to this theory, we have scripts (think acting) that help us schedule actions.  Another theory that attempts to explain how we know what to do is the affordences theory of Gibson. A chair affords sitting, for example, and that is how we know what to do with chairs.  The two theories are not quite at odds with each other: they are just trying to explain similar phenomena at different granularities.

The micro decisions seqsee makes in choosing what codelets should be created next occurs via an affordences-like mechanism. Groups afford extending, for example.  Even meta thoughts afford.  The thought \emph{I am in a rut} affords destroying weak groups and trying less likely possibilities like exploring if the sequence is interlaced.

Seqsee does not currently use scripts.  It is conceivable that something similar may be called for.

\subsection{The Stream}
\label{sec:stream}

Where all this comes together is the stream of thought.  It is the star player in the central cognitive loop.  Seqsee has objects (in the workspace) and thoughts about those objects in the stream.  Unlike the highly parallel buzz of activity of the codelets, there is something slightly more linear about the stream.  When I describe the stream, people sometimes have the sense of one thought leading to the next leading to the next---the $n+1^{\textrm{th}}$ thought being caused somehow by the $n^\textrm{th}$ thought.  That is not how Seqsee's stream works.  I must describe it more carefully here.

One way to start describing it is to say where the $n+1^{\textrm{th}}$ thought can come from.  It can of course come  from the immediate prior thoughts but more importantly it can come from any codelet whatsoever.  Consider the hypothetical case where seqsee is solving five problems in parallel.  There are several codelets toiling away on each problem.  A thought in the stream corresponds to focusing attention on something for a short period of time.  In principle it could happen that the thoughts that pass through the stream are perfectly interlaced: the first, the sixth, the 11th thoughts coming from the first problem and so forth.  In this sense, then, the stream is just a list of all thoughts happening anywhere in the system, just sorted by time.  So much for linearity.

If the stream is almost merely a birth-registery of attention-fixations what, if anything, does it buy us? What it gives us is temporality.  Seqsee keeps in memory the fringes of the last few thoughts, and these decay with time.  If the fringe of a new thought intersects with the fringe of one of the more recent thoughts from anywhere in the system seqsee's similarity-ears get perked.  In the hypothetical five problem case we would see very strong recency effects and interference from seqsee.  It is as if the stream is distributed throughout the "brain" of seqsee, but whenever a thought occurs anywhere its echo reverberates throughout, potentially influencing for a short time the processing everywhere.

This is not a completely unmotivated move on my part.  A somewhat similar mechanism is used by \citeN{Dennett:Consciousness} in what he calls the \emph{Joycean machine}---a more or less serial machine simulated imperfectly on parallel neural hardware.

More pragmatically, using the stream has made seqsee much smarter (as compared to Seqsee 12 months ago when there was no stream).  In a sequence like \emph{1 1 1 1 17 2 2 2 2 18} the way seqsee discovers the 17-18 relationship is by being lucky and thinking about the 17 and 18 somewhat close together in time.  Because of all the hubbub of the parallel codelet activity such lucky episodes are guaranteed to happen.  Without the stream seeing the 17 -- 18 relationship was much more unlikely.  With a bond scout like mechanism where objects are randomly chosen it is necessary to bias against choosing objects distant from each other because such random long-distance checking is statistically less likely to produce useful results.  So even though the 17-18 relationship stands a chance of being discovered it is not predestined.


%\bibliographystyle{apalike}
%\bibliography{merged}
%\end{document}


\section{Implications of success}
\section{A tentative time-frame}
\bibliographystyle{chicago}
\bibliography{merged}
\printindex
\end{document}
