\documentclass{article}

\begin{document}
\title{Meeting Doug}
\date{Sep 19, 2005}
\author{Abhijit Mahabal}
\maketitle

I am far more disorganized today, partly because of the large range of things I have been thinking about, and don't quite know where to start.

One possible place could be me trying to do a dry run of a few sequences Seqsee should handle. If we take this route I'd use the following sequences (no surprises in the sequences, I promise):

\begin{itemize}
\item 1 2 3 4 5 6 7 8 9 10 11 12 13 14 15 16$\ldots$

Even this sequence is not without interest: you probably caught on really early as to what this was. I think this sense of ``I got it'' is an important thing to model, and this sequence is one place to start. 

\item 1 17 2 17 3 17 4 17$\ldots$

and

\item 1 17 1 2 17 1 2 3 17$\ldots$



\end{itemize}

Already in these sequences a most of the architecture would get a full workout. Issues that I expect would come up include:
\begin{itemize}
\item What are the thoughts in the stream?
\item Why we need a stream, what having a stream buys us. \footnote{If I look at myself working, I do see pressures pulling in several directions, but there is also a very pronounced tendency to follow a chain of thought a while(between 1 second and maybe 20 seconds, but sometimes, in a vague way, an hour or longer, though there I suspect its several related threads starting off one after the other, somewhat like the process of searching a wallet).}
\item What are some families of thoughts
\end{itemize}

Another place to start, which I'd prefer, is asking Doug examples of what some Seqsee codelets could do, so as to get a sense of the ``size'' of these things.

\end{document}
